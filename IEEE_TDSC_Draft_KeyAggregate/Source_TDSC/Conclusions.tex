\section{Conclusions}
\label{sec:conclusions}

% In this paper, we have proposed a provably secure scalable key-aggregate cryptosystem (KAC) online data sharing on the cloud. The main aim of this cryptosystem is to allow data owners to delegate decryption rights to any subset encrypted files shared on the cloud using a constant-sized \emph{aggregate key}. We have described the construction of a basic KAC unit that allows a single data owner to share her data online using constant sized ciphertext messages and linear-sized public parameters. We have proved this basic scheme to be CPA secure, and have also demonstrated how it may be suitably modified to achieve CCA security as well. To the best of our knowledge, this is the first KAC in the cryptographic literature that achieves CCA security. We have then demonstrated how this system can be actually deployed on the cloud securely and efficiently to accommodate multiple data owners. We laid particular focus on the performance, efficiency and scalability of the system. Finally, we have proposed a novel hierarchical KAC that allows a data owner to store her data in a hierarchical fashion on the cloud. Decryption can be performed via local aggregation at each hierarchy level. We have presented simulation results on an actual implementation of the system using bilinear Tate pairings, that showed that the proposed hierarchical KAC outperforms standard hierarchical key-based cryptosystems on performance and efficiency. We have also laid down guidelines on the best hierarchical construction for optimum performance in the KAC framework. We have also discussed possible application areas where KAC could be practically deployed.

% In this paper, we have proposed a secure and dynamic key aggregate encryption scheme for online data sharing. Our scheme allows data owners to delegate users with access rights to multiple ciphertext classes using a single decryption key that combines the decrypting power of individual keys corresponding to each ciphertext class. Unlike existing key aggregate schemes that are static in their access right delegation policies, our scheme allows data owners to dynamically revoke one or more users' access rights without having to change either the public or the private parameters/keys. The use of bilinear pairings over additive elliptic curve subgroups in our scheme helps achieve massive reductions in key and ciphertext sizes over existing schemes that use multiplicative groups. We pointed out that a possible criticism of this scheme is that the number of classes is pre-defined to some fixed $n$. To deal with this issue, we next proposed a generalized two-level construction of the basic scheme that runs $n_1$ instances of the basic scheme in parallel, with each instance handling key aggregation for $n_2$ ciphertext classes. This scheme provides two major advantages. First of all, it allows dynamic extension of ciphertext classes by registering of new public key-private key pairs without affecting other system parameters. Secondly, it provides a wide range of choices for $n_1$ and $n_2$ that allows efficient utilization of ciphertext classes while also achieving optimum space and time complexities. Finally, we extend the generalized scheme to allow the use of popular and efficiently implementable bilinear pairings in literature such as Tate Pairings that operate on multiple elliptic curve subgroups instead of one. Each of the three proposed schemes have been proven to be semantically secure. Experimental studies have demonstrated the superiority of our proposed scheme over existing ones in terms of key size as well as efficient utilization of ciphertext classes. A possible future work is to make the proposed schemes secure against chosen ciphertext attacks.