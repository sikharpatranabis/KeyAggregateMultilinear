\section{Related Work}
\label{sec:relwork}

In this section we present a brief overview of public and private key cryptographic schemes in literature for secure online data sharing. While many of them focus on key aggregation in some form or the other, very few have the ability to provide constant size keys to decrypt an arbitrary number of encrypted entities. One of the most popular techniques for access control in online data storage is to use a pre-defined hierarchy of secret keys \cite{ateniese2012provably} in the form of a tree-like structure, where access to the key corresponding to any node implicitly grants access to all the keys in the subtree rooted at that node. A major disadvantage of hierarchical encryption schemes is that granting access to only a selected set of branches within a given subtree warrants an increase in the number of granted secret keys. This in turn blows up the size of the key shared. Compact key encryption for the symmetric key setting has been used in \cite{benaloh2009patient} to solve the problem of concisely transmitting  large number of keys in the broadcast scenario. However, symmetric key sharing via a secured channel is costly and not always practically viable for many applications on the cloud. Proxy re-encryption is another technique to achieve fine-grained access control and scalable user revocation in unreliable clouds \cite{ateniese2006improved}. However, proxy re-encryption essentially transfers the responsibility for secure key storage from the delegatee to the proxy and is susceptible to collusion attacks. It is also important to ensure that the transformation key of the proxy is well protected, and every decryption would require a separate interaction with the proxy, which is inconvenient for applications on the cloud.

The authors of \cite{chu2014key} proposes an efficient scheme, namely KAC, that allows secure and efficient sharing of data on the cloud. The scheme is a public-key cryptosystem that uses constant size ciphertexts such that efficient delegation of decryption rights for any set of ciphertexts are possible. When a user demands for a particular subset of the available classes of data, the data owner computes an aggregate key which integrates the power of the individual decryption keys corresponding to each class of data. KAC as proposed in \cite{chu2014key} suffers from three major drawbacks, each of which we address in this paper. First of all, the security assumption of KAC seems to be the Bilinear Diffie Hellman Exponent (BDHE) assumption \cite{miller1986use}; however no concrete proofs of semantic security are provided by the authors in \cite{chu2014key}. Secondly, with respect to user access rights, KAC is a static scheme in the sense that once a user is in possession of the aggregate key corresponding to a subset of files from data owner, the owner cannot dynamically revoke the permission of the client for accessing one or more updated files. Since dynamic changes in access rights is extremely common in online data storage, this scenario needs to be tackled. Finally, the public key extension of KAC proposed in \cite{chu2014key} is extremely cumbersome and resource consuming since registration of each new public key-private key pair requires the number of classes to be extended by the original number of classes.  





