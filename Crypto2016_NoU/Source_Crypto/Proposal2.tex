\section{KAC Using Symmetric Multilinear Maps}
\label{sec:proposal2}

In this section, we present a KAC construction based on traditional symmetric multilinear maps. We use the same idea presented in the earlier construction, that is, we use a symmetric multilinear map to ensure that the necessary parameters can be derived from a small number of public elements in the source group of the map. In this construction, the parameter $Y_i=g^{\alpha^i}_m$, while $X_j=g^{\alpha^{2^j}}_1$. However, unlike in the asymmetric setting where the same elements cannot be paired together, in the symmetric setting one could pair $X_{m-1}$ with itself, and then pair it with $g_1$ $m-2$ times, to obtain $Y_{N+1}$. To overcome this we use a technique proposed in \cite{boneh2014low} that restricts the bit representations of all identities in $\mathcal{ID}$ to a single Hamming weight class. This actually allows computing the necessary $Y_i$ without leaking the value of $Y_{N+1}$. The idea is illustrated in the following discussion.

\subsubsection{The Basic Idea.} Let $Y_i=g^{\alpha^i}_{m-1}$ and $\hat{Y}_i=g^{\alpha^i}_l$ where $l\leq m$. Set $X_j=g^{\alpha^{(2^j)}}_1$ for $i=0,1,\cdots,m$. Further, let $HW(i)$ denote the Hamming weight of the bit representation of $i$. We now make the following claims.

\subsubsection{Claim 5.1.} One can compute $g^{\alpha^i}_{HW(i)}$ for $1\leq i\leq 2^m-2$. In particular, one can compute $\hat{Y}_i$ for $1\leq i\leq 2^m-2$ such that $HW(i)=l$.\\

\noindent\textbf{Proof.} Compute $g^{\alpha^i}_{HW(i)}$ by pairing together all $X_j$ such that the $j$th bit of $i$ is 1. Since $i$ has at most $m$ bits, the necessary $X_j$ are available. 

% \subsubsection{Claim 5.2.} One can compute $Y_{i}$ for $1\leq i\leq 2^m-2$.\\
% 
% \noindent\textbf{Proof.} Compute $g^{\alpha^i}_{HW(i)}$ by Claim 5.1. (and pair it with $g_{m-HW(i)-1}$ if $HW(i)\leq m-2$). Note that for all $i$ such that $1\leq i\leq 2^m-2$, $HW(i)\leq m-1$.

\subsubsection{Claim 5.2.} One can compute $Y_{i}$ and $Y_{2^m-1-i}$ for $1\leq i\leq 2^m-2$ such that $HW(i)=l$.\\

\noindent\textbf{Proof.} Note that for all $i$ such that $1\leq i\leq 2^m-2$, $HW(i)\leq m-1$. Hence, one can compute $g^{\alpha^i}_{HW(i)}$ by Claim 5.1 and then pair it with $g_{m-HW(i)-1}$ (if $HW(i)\leq m-2$) to obtain $Y_{i}$.  Also, compute $g^{\alpha^{2^m-1-i}}_{HW(2^m-1-i)}$ as per Claim 5.1. Note that $HW(2^m-1-i)=m-l$ if $HW(i)=l$. Thus, we basically computed $g^{\alpha^{2^m-1-i}}_{m-l}$. Then, we pair it with $g_{l-1}$ (obtained by pairing $g_1$ ($l-1$) times) to obtain $Y_{2^m-1-i}$. 

\subsubsection{Claim 5.3.}  One can compute $Y_{2^m-1-v+i}$ for $1\leq i,v\leq 2^m-2$, $i\neq v$ where $HW(i)=HW(v)=l$.\\

\noindent\textbf{Proof.} Let $T_1$ denote the set of these bit positions that are $1$ in the bit representation of $i$, and $T_2$ denote the set of bit positions that are $1$ in the bit representation of $2^m-1-v$. Clearly, $|T_1|=l$ and $|T_2|=m-l$. Now, note that that $T_1\bigcap T_2=\phi$ iff $i=v$ which is not allowed. So $\exists j'\in T_1\bigcap T_2$. Then, we can write
\begin{equation}
{2^m-1-v+i}=\left(\sum_{j\in T_1\backslash\{j'\}}2^j\right) + \left(\sum_{j\in T_2\backslash\{j'\}}2^j\right) + 2^{j'+1}\nonumber
\end{equation}
\noindent Note that this is a sum of $m-1$ powers of two. This in turn allows us to compute
\begin{equation}
 Y_{2^m-1-v+i}=e\left(\{X_j\}_{j\in T_1\backslash\{j'\}},\{X_j\}_{j\in T_2\backslash\{j'\}},X_{j'+1}\right)\nonumber
\end{equation}
which is a pairing of $(m-1)$ $X_j$ terms, as desired.

\subsubsection{Assumption 5.4.} For simplicity, we assume in the forthcoming discussion that our plaintext messages are embedded as elements in the group $\mathbb{G}_{m+l-1}$. For relaxations, refer Appendix \ref{app_sec:relaxation}.

% \subsection{A Basic Construction}
% \label{subsec:construction_2}
% 
% We now present the basic version of our second construction for KAC using symmetric multilinear maps. Recall that \textbf{SetUp}$'(1^\lambda,m)$ sets up an $m$-linear map with groups of prime order $q$ ($q$ being a $\lambda$ bit prime) and the target group $\mathbb{G}_{m}$. Our second identity-based KAC consists of the following algorithms.\\
% 
% \noindent\textbf{SetUp}$(1^{\lambda},(m,l))$: Set up the KAC system for $\mathcal{ID}$ consisting of all $m$ bit class identities with Hamming weight $l$, that is $N=|\mathcal{ID}|=\binom{m}{l}$. Since $1\leq l\leq m-1$, we have $N\leq 2^{m-2}$. Let $param'\leftarrow SetUp'(1^{\lambda},m+l-1)$ be the public parameters for a symmetric multilinear map, with $\mathbb{G}_{m+l-1}$ being the target group. Choose a random $\alpha\in \mathbb{Z}_q$. Set $X_j=g^{\alpha^{(2^j)}}_{1}$ for $0\leq j\leq m$.  Output the public parameter tuple $param$ as
% \begin{equation}
%  param = (param',\{X_j\}_{j\in\{0,\cdots,m\}})\nonumber
% \end{equation}
% \noindent Discard $\alpha$ after $param$ has been output.\\
% 
% \noindent \textbf{KeyGen}(): Randomly pick $\gamma\in \mathbb{Z}_q$. Set $msk=\gamma$ and $PK=g^{\gamma}_{l}$. Output the tuple $(msk,PK)$.\\
% 
% \noindent \textbf{Encrypt}$(param,PK,i,\mathcal{M})$: Take as input a message $\mathcal{M} \in \mathbb{G}_{m+l-1}$ belonging to class $i \in \mathcal{ID}$. Randomly choose $t \in\mathbb{Z}_q$. Recall that $\hat{Y}_i=g^{\alpha^i}_{l}$ and can be computed for $i\in\mathcal{ID}$ as per Claim 5.1. Output the ciphertext $\mathcal{C}$ as 
% \begin{equation}
%  \mathcal{C}=(g^t_{l},(PK.\hat{Y}_i)^{t},\mathcal{M}.g^{t\alpha^{(2^m-1)}}_{m+l-1})\nonumber
% \end{equation}
% 
% \noindent \textbf{Extract}$(param,msk,\mathcal{S})$: Let $msk=(msk_1,msk_2)$. For the input subset of data class indices $\mathcal{S}$, the aggregate key is computed as 
% \begin{equation}
%  K_{\mathcal{S}} = \prod_{v\in\mathcal{S}}\left({Y_{2^m-1-v}}\right)^{msk_1}\nonumber
% \end{equation} 
% \noindent Recall that $Y_{2^m-1-v}$ can be computed as per Claim 5.3 for $j\in\mathcal{ID}$. Note that this is equivalent to setting $K_{\mathcal{S}}$ to $\prod_{v\in\mathcal{S}}{\left(g^{msk}_{m-1}\right)}^{\alpha^{2^m-1-v}}$.\\
%  
% \noindent\textbf{Decrypt}$(param,\mathcal{C},i,\mathcal{S},K_{\mathcal{S}})$: If $i\notin\mathcal{S}$, output $\bot$. Otherwise, use the results from Claims 5.3 and 5.2 to set
% \begin{equation}
% a_{\mathcal{S}}=\left(\prod_{v\in\mathcal{S},v\neq i}Y_{2^m-1-v+i}\right) \text{ and } b_{\mathcal{S}}=\left(\prod_{v\in\mathcal{S}}Y_{2^m-1-v}\right)\nonumber
% \end{equation}
% Let $\mathcal{C}=(c_0,c_1,c_2)$. Output the decrypted message as  
% \begin{eqnarray} 
% \hat{\mathcal{M}}&=&c_2\frac{{e}(K_{\mathcal{S}}.a_{\mathcal{S}},c_0)}{{e}(b_{\mathcal{S}},c_1)} \nonumber
% \end{eqnarray}
% 
% \subsubsection{Correctness.} The correctness proof is similar to that of the construction proposed in Section \ref{subsec:construction}. Refer Appendix \ref{app_sec:correctness_2} for details.\\
% 
% \noindent Finally, we comment on the choice of $m$ and $l$. Let $N=|\mathcal{ID}|=2^{\lambda}$ be the number of classes our proposed KAC wishes to handle. If we wish to minimize the value of $m$, we may set $m=\lambda+\lceil(\log_2\lambda)/2\rceil+1$ and $l=\lfloor m/2\rfloor$. But if we wish to minimize the degree of multilinearity, then we must set $m\approx 1.042(\lambda+(\log_2\lambda)/2)$ and $l\approx 0.398(\lambda+(\log_2\lambda)/2)$, leading to a total multilinearity requirement of $1.44(\lambda+(\log_2\lambda)/2)-1$ \cite{boneh2014low}.
% 
% % \subsubsection{Implementation Nuances.} As in the earlier construction, the system administrator must herself set up the multilinear map framework. Also, the ciphertext and the aggregate key must not leak any important information and hence need to be appropriately randomized. This implies that Kilian-style randomization parameters must be included for the groups $\mathbb{G}_{l}$ and $\mathbb{G}_{m-1}$. We now look into the security of our second identity-based KAC construction.
% 
% % \subsection{The Complexity Assumption}
% % \label{subsec:complexity_2}
% % 
% 
% 
% \subsection{Security}
% \label{subsec:security2}
% 
% We first briefly state the complexity assumption that is to be used to prove the security of second KAC scheme.
% 
% \subsubsection{The Multilinear Diffie-Hellman Exponent Assumption.} Let $param'$ is generated by \textbf{SetUp}$'(1^{\lambda},m+l-1)$. Choose $\alpha \in \mathbb{Z}_q$ at random (where $q$ is a $\lambda$-bit prime), and let $X_j=g^{\alpha^{(2^j)}}_{1}$ for $0\leq j \leq m$. Choose a random $t\in\mathbb{Z}_q$, and let $V=g^{t}_{l}$. The decisional $(m,l)$-Multilinear Diffie Hellman Exponent~(MDHE) problem as defined as follows. Given the tuple $(param',\{X_j\}_{j\in\{0,\cdots,m\}},V,Z)$, distinguish if $Z$ is $g^{t\alpha^{(2^m-1)}}_{m+l-1}$ or a random element of $\mathbb{G}_{m+l-1}$.
% 
% \subsubsection{Definition 4.5.} The decisional $(m,l)$-Multilinear Diffie-Hellman Exponent assumption holds for {SetUp}$'$ if, for any polynomial $m$ and a probabilistic poly-time algorithm $\mathcal{A}$, $\mathcal{A}$ has negligible advantage in solving the $m$-Multilinear Diffie-Hellman Exponent problem.\\
% 
% We next state the folliwng theorem for the security of our symmetric key based KAC construction.
% 
% \subsubsection{Theorem 4.6.} \textit{Let \textbf{Setup}$'$ be the setup algorithm for a symmetric multilinear map, and let the decisional $(m,l)$-Multilinear Diffie-Hellman Exponent assumption holds for {SetUp}$'$. Then our proposed construction of KAC for $N$ data classes presented in Section \ref{subsec:construction_2} is non-adaptively CPA secure for $N=\binom{m}{l}$.}
% 
% 
% % \subsubsection{CCA Security.} 
% 
% \subsubsection{Extensions.} The KAC construction can be extended for public-key based aggregate key distribution to multiple data users using techniques similar to those in Section \ref{subsec:multiuserKAC}. For details, refer Appendix \ref{subsec:multiuserKAC_2}. Extensions for multiple data owners also follow from discussions in Section \ref{subsec:multiownerKAC}. Finally, we can easily extend this KAC construction for non-adaptive CCA security with full collusion resistance. For the detailed construction and security proof, refer Appendix \ref{app_sec:CCAsecure2}.

\subsection{An Extended KAC Construction with Broadcast Aggregate Keys}
\label{subsec:multiuserKAC_2}

We present an extended KAC construction using symmetric multilinear maps with the ability to broadcast aggregate keys for arbitrarily large subsets of data classes $\mathcal{S}$ to arbitrarily large subsets of data users $\hat{\mathcal{S}}$. Our scheme handles $N$ data classes and $N$ users, as in the extended KAC construction presented in Section \ref{subsec:multiownerKAC}.
% 
\subsubsection{Construction.} Let $N=2^m-2$ and $Setup'$ sets up a symmetric multilinear map. Recall that $Y_i=g^{\alpha^i}_{m-1}$ and $\hat{Y}_i=g^{\alpha^i}_l$, where $1\leq i \leq N$ and $l\leq m$. Our scheme can handle $N$ data classes and $N$ data users.\\\\
% 
\noindent\textbf{SetUp}$(1^{\lambda},m)$: \noindent\textbf{SetUp}$(1^{\lambda},(m,l))$: Set up the KAC system for $\mathcal{ID}$ consisting of all $m$ bit class identities with Hamming weight $l$, that is $N=|\mathcal{ID}|=\binom{m}{l}$. Since $1\leq l\leq m-1$, we have $N\leq 2^{m-2}$. Let $param'\leftarrow SetUp'(1^{\lambda'},m+l-1)$ be the public parameters for a symmetric multilinear map, with $\mathbb{G}_{m+l-1}$ being the target group and $\lambda'$ is the adequate group size parameter for achieving $\lambda$ bit security. Choose a random $\alpha\in \mathbb{Z}_q$. Set $X_j=g^{\alpha^{(2^j)}}_{1}$ for $0\leq j\leq m$.  Output the public parameter tuple $param$ as
\begin{equation}
 param = (param',\{X_j\}_{j\in\{0,\cdots,m\}})\nonumber
\end{equation}
\noindent Discard $\alpha$ after $param$ has been output.\\\\
% 
\noindent\textbf{OwnerKeyGen}(): Randomly pick $\gamma_1, \gamma_2, \gamma_3\in \mathbb{Z}_q$. Set the master secret key $msk=(\gamma_1,\gamma_2)$, the public key $PK=(g^{\gamma_1}_{l},g^{\gamma_2}_{l})$ and the distribution key $dsk=\gamma_3$. Output the tuple $(msk,PK,dsk)$.\\\\
% 
\noindent\textbf{Encrypt}$(param,PK,dsk,i,\mathcal{M})$: Take as input a message $\mathcal{M} \in \mathbb{G}_{m+l-1}$ belonging to class $i \in \mathcal{ID}$. Randomly choose $t \in\mathbb{Z}_q$. Output the ciphertext $\mathcal{C}$ as 
\begin{equation}
 \mathcal{C}=\left(g^{t}_{l},PK^{t-dsk}_1,(PK_1.\hat{Y}_i)^{t},\mathcal{M}.g^{t\alpha^{(2^m-1)}}_{m+l-1}\right)\nonumber
\end{equation}

\noindent\textbf{UserKeyGen}$(param,msk,\hat{i})$: Let $msk=(msk_1,msk_2)$. Output the secret key for data user $\hat{i}$ as
\begin{equation}
 d_{\hat{i}}=Y^{msk_2}_{\hat{i}}\nonumber 
\end{equation}


\noindent \textbf{Extract}$(param,msk,\mathcal{S})$: Let $msk=(msk_1,msk_2)$. For the input subset of data class indices $\mathcal{S}$, the aggregate key is computed as 
\begin{equation}
 K_{\mathcal{S}} = \prod_{v\in\mathcal{S}}\left({Y_{2^m-1-v}}\right)^{msk_1}\nonumber
\end{equation} 
\noindent Recall that $Y_{2^m-1-v}$ can be computed as per Claim 5.3 for $j\in\mathcal{ID}$. Note that this is equivalent to setting $K_{\mathcal{S}}$ to $\prod_{v\in\mathcal{S}}{\left(g^{msk}_{m-1}\right)}^{\alpha^{2^m-1-v}}$.\\\\

\noindent\textbf{Broadcast}$(param,K_{\mathcal{S}},\hat{\mathcal{S}},PK,dsk)$: Broadcasts the aggregate key $K_{\mathcal{S}}$ to all users in $\hat{\mathcal{S}}$ as follows. Let $PK=(PK_1,PK_2)$. Randomly choose $\hat{t}\in\mathbb{Z}_q$ and set
\begin{equation}
b_{\hat{\mathcal{S}}}=\left(\prod_{\hat{v}\in\hat{\mathcal{S}}}Y_{2^m-1-\hat{v}}\right)\nonumber
\end{equation}
\noindent Output 
\begin{equation}
K_{\left(\mathcal{S},\hat{\mathcal{S}}\right)}=\left(g^{\hat{t}}_{l},\left(g^{msk_2}_{m-1}.b^{\hat{t}}_{\hat{\mathcal{S}}}\right),\mathcal{K}\right)\nonumber
\end{equation}
\noindent where 
\begin{equation}
\mathcal{K} = \left(\left(g^{\hat{t}\alpha^{(2^m-1)}}_{m+l-1}\right).\left(e(K_{\mathcal{S}},g^{dsk}_{l})\right)\right)\nonumber
\end{equation}
\\\\
%  
\noindent\textbf{Decrypt}$(param,\mathcal{C},K_{\left(\mathcal{S},\hat{\mathcal{S}}\right)},i,\hat{i},d_{\hat{i}},\mathcal{S},\hat{\mathcal{S}})$: If $i\notin\mathcal{S}$ or $\hat{i}\notin\hat{\mathcal{S}}$, output $\bot$. Otherwise, set
\begin{eqnarray}
 a_{\hat{\mathcal{S}}}=\left(\prod_{\hat{v}\in\hat{\mathcal{S}},\hat{v}\neq \hat{i}}Y_{2^m-1-\hat{v}+\hat{i}}\right)&\text{ , }&
 a_{\mathcal{S}}=\left(\prod_{v\in\mathcal{S},v\neq i}Y_{2^m-1-v+i}\right)\nonumber\\ 
 \text{and }b_{\mathcal{S}}&=&\left(\prod_{v\in\mathcal{S}}Y_{2^m-1-v}\right)\nonumber
\end{eqnarray}
\noindent Let $\mathcal{C}=(c_0,c_1,c_2,c_3)$ and $K_{\left(\mathcal{S},\hat{\mathcal{S}}\right)}=(\hat{k}_0,\hat{k}_1,\hat{k}_2)$. Output the decrypted message as  
\begin{eqnarray} 
\hat{\mathcal{M}}&=&c_3.\hat{k}_2.\left(\frac{e(b_{\mathcal{S}},c_1){e}(a_{\mathcal{S}},c_0)}{{e}(b_{\mathcal{S}},c_2)}\right).\left(\frac{e(d_{\hat{i}}.a_{\hat{\mathcal{S}}},\hat{k}_0)}{e(Y_{\hat{i}},\hat{k}_1)}\right) \nonumber
\end{eqnarray}

\noindent Correctness of this scheme may be easily proven. We introduce the complexity assumption for the proof of security of this scheme here.

\subsubsection{The Extended Multilinear Diffie-Hellman Exponent Assumption.} Let $param'$ is generated by \textbf{SetUp}$'(1^{\lambda'},m+l-1)$. Choose $\alpha \in \mathbb{Z}_q$ at random (where $q$ is a $\lambda'$-bit prime), and let $X_j=g^{\alpha^{(2^j)}}_{1}$ for $0\leq j \leq m$. Choose random $t,\hat{t}\in\mathbb{Z}_q$, and let $(V_1,V_2)=\left(g^{t}_{l},g^{\hat{t}}_{l}\right)$. The decisional $(m,l)$-Extended Multilinear Diffie Hellman Exponent~(EMDHE) problem as defined as follows. Given the tuple 
\begin{equation}
\left(param',\{X_j\}_{j\in\{0,\cdots,m\}},(V_1,V_2),(Z_1,Z_2)\right)\nonumber
\end{equation}
\noindent distinguish if $(Z_1,Z_2)$ is $\left(g^{t\alpha^{(2^m-1)}}_{m+l-1},g^{\hat{t}\alpha^{(2^m-1)}}_{m+l-1}\right)$ or a random element of $\mathbb{G}_{m+l-1}\times\mathbb{G}_{m+l-1}$.

\subsubsection{Definition 5.5.} The decisional $(m,l)$-Extended Multilinear Diffie-Hellman Exponent (EMDHE) assumption holds for {SetUp}$'$ if, for any polynomial $m$ and a probabilistic poly-time algorithm $\mathcal{A}$, $\mathcal{A}$ has negligible advantage in solving the $(m,l)$-EMDHE problem.

\subsubsection{Theorem 5.6.} \textit{Let \textbf{Setup}$'$ be the setup algorithm for an symmetric multilinear map, and let the decisional $(m,l)$-EHDHE assumption holds for {SetUp}$'$. Then the extended multi-user KAC supporting $N$ data classes and $N$ data users is non-adaptively CPA secure, where $N=\binom{m}{l}$.}\\

\noindent The detailed proof of Theorem 5.6. is very similar to that of Theorem 3.8 in Section \ref{subsec:security1_1} and is hence avoided. Finally, we can easily extend this KAC construction for non-adaptive CCA security with full collusion resistance. We present a basic CCA secure KAC construction using symmetric multilinear maps in Appendix \ref{app_sec:CCAsecure2}.












