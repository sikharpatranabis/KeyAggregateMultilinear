\section{Applications of KAC}
\label{sec:applications}

The key-aggregate encryption systems described in this paper are primarily meant for data sharing on the cloud. In this section, we point out some specific applications in which KAC proves to be a very efficient solution.

\subsubsection{Online Collaborative Data Sharing.} The foremost application of KAC is in secure data sharing for collaborative applications. Applications such as Google Drive \cite{googledrive} and Dropbox \cite{dropbox} allow users to share their data on the cloud and delegate access rights to multiple users to specific subsets of their whole data. Even government and corporate organizations require secure data sharing mechanisms for their daily operations. KAC can be easily set up to function on top of standard data sharing applications to provide security and flexibility. Data classes may be viewed as folders containing similar files. The fact that our proposed KAC is identity based means that each folder can have its own unique ID chosen by the data owner. Also, the fact that the ciphertext overhead is only logarithmic in the number of data classes implies that space requirement for any data owner is optimal. Finally, the aggregate key also has low overhead and can be transmitted via a secure channel such as a password protected mail service. Since KAC is easily extensible to multiple data owners, the system is practically deployable for a practical data sharing environment. The other advantage of KAC is that once a system is setup with a set of multilinear maps and public parameters, the same setup with the same set of public parameters can be reused by multiple teams within the same organization. Since data owned by each individual owner is insulated from access by users who do not have the corresponding aggregate key, and each data owner has her own tuple of public, private and authentication keys, a single KAC can support multiple data sharing units, while guaranteeing the same underlying security. This saves the cost of setting up new multilinear maps and public parameters each time. 

\subsubsection{Distribution of Product License and/or Activation Keys.} Suppose a company owns a number of products, and intends to distribute the license files (or activation keys) corresponding to these to different users. The KAC framework allows them to put these keys on the cloud in an encrypted fashion, and distribute an aggregate key corresponding to the license files for multiple products to legally authenticated customers as per their requirements. The legal authentication comes from the fact the user who buys multiple products from the company is given the authentication key and the aggregate key that allows her to decrypt the license file for each product. Since both these keys are of constant size, distributing  these to users is easier than providing a separate license file to each user.

\subsubsection{Patient controlled encryption (PCE).} Patient controlled encryption~(PCE) is a recent concept that has been studied in the literature \cite{benaloh2009patient}. PCE allows a patient to upload her own medical data on the cloud and delegate decryption rights to healthcare personnel as per her requirement. KAC acts as an efficient solution to this problem by allowing patients to define their own hierarchy of medical data and delegate decryption rights to this data to different specialists/medical institutions using aggregate keys in an efficient fashion. Given the multitude of sensitive digital health records existent in today's world, storing this data in local/personal machines is not a viable solution and the cloud seems the best alternative. KAC thus provides a two-way advantage in this regard. Not only does it allow people from across the globe to store their health data efficiently and safely, but also allows them to envisage the support of expert medical care from across the globe. 


