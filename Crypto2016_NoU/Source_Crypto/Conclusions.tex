\section{Conclusions and Open Problems}
\label{sec:conclusions}

We presented the first fully identity-based key-aggregate cryptosystem~(KAC) for access delegation for arbitrarily large subsets of data classes shared online, among any number of authorized data users. We proposed three different $O(\log N)$-way multilinear map-based constructions that support $N$ data classes and $N$ data users, with low overhead for ciphertexts, aggregate keys and all public and private parameters. We proved the security and collusion resistance of each of the schemes under different security assumptions, and also discussed the various implementation nuances and trade-offs associated with them. For broadcasting data access rights among multiple users, we showed how to efficiently combine the stand-alone KAC constructions with broadcast encryption schemes. Each of our constructions give rise to full-fledged public key based data sharing systems with collusion resistance against any number of colluding parties. We discussed potential applications for such data sharing schemes, including collaborative research and healthcare.

We leave as an open problem the question of building KAC constructions that are secure against adaptive adversaries in the standard model. Another interesting problem is to build efficiently revocable cryptosystems for data sharing with traitor tracing properties. In particular, it would be interesting to build a cryptosystem that allows revoking a user's access rights without the need for re-encrypting all the shared data. Finally, a possible future work is to present an actuaal instantiation of our proposed scheme using the graph-induced multilinear map candidate \cite{gentry2015graph} while avoiding weak trapdoor attacks. 