\section{CCA Secure KAC Using Symmetric Multilinear Maps}
\label{app_sec:CCAsecure2}

We now demonstrate how to extend the identity-based KAC construction using multilinear maps to obtain non-adaptive chosen ciphertext security. We again make use of the signature scheme coupled with the collusion resistant hash function that we introduced in Section \ref{subsec:construction_CCA}. In this CCA secure construction, we force the class index value $i$ to be strictly less than $2^m-2$ instead of $2^m-1$ in the CPA secure construction.

\noindent\textbf{SetUp}$(1^{\lambda},m)$: Same as in Section \ref{subsec:construction_2}.\\
 
\noindent \textbf{KeyGen}(): Same as in Section \ref{subsec:construction_2}.\\
 
\noindent \textbf{Encrypt}$(PK,i,\mathcal{M})$: Run the $SigKeyGen$ algorithm to obtain a signature signing key $K_{SIG}$ and a verification key $V_{SIG} \in \mathbb{Z}_q$. Randomly choose $r\in\mathbb{Z}_q$ and let $t'=t+r \in\mathbb{Z}_q$. Compute 
\begin{equation}
 \mathcal{C}'=(g^r_{l},(PK.Y_i.g^{V_{SIG}}_{m-1})^{t'},\mathcal{M}.g^{t'\alpha^{(2^m-1)}}_{m+l-1})\nonumber
\end{equation} 
\noindent and output the ciphertext as
\begin{eqnarray}
 \mathcal{C}&=&(\mathcal{C}',Sign(\mathcal{C}',K_{SIG}),V_{SIG}) \nonumber
\end{eqnarray} 
 
\noindent \textbf{Extract}$(msk,\mathcal{S})$: Same as in Section \ref{subsec:construction_2}.\\
 
\noindent \textbf{Decrypt}$(\mathcal{C},i,\mathcal{S},K_{\mathcal{S}},U)$: Let $\mathcal{C}=((c_0,c_1,c_2),\sigma,V_{SIG})$. Verify that $\sigma$ is a valid signature of $(c_0,c_1,c_2)$ under the key $V_{SIG}$. If not, output $\bot$. Also, if $i\notin\mathcal{S}$, output $\bot$. Otherwise, set 
\begin{eqnarray}
  SIG_{\mathcal{S}} &=& \prod_{v\in\mathcal{S}}g^{V_{SIG}}_{m-1}\nonumber\\
  a_{\mathcal{S}} &=& \prod_{v\in\mathcal{S},v\neq i}Y_{2^m-1-v+i}\nonumber\\
  b_{\mathcal{S}} &=& \prod_{v\in\mathcal{S}}Y_{2^m-1-v}\nonumber  
\end{eqnarray}
% \noindent Note that these can be computed as $1\leq i,v \leq N-1(=2^m-2)$. \emph{This is precisely why we allow only $N-1$ data classes}.  
\noindent Next, pick a random $w\in \mathbb{Z}_q$ and set
 \begin{eqnarray}
  \hat{d_1} &=& (K_{\mathcal{S}}.SIG_{\mathcal{S}}.a_{\mathcal{S}}.(PK.Y_i.g^{V_{SIG}}_{m-1})^w)\nonumber\\
  \hat{d_2} &=& (b_{\mathcal{S}}.g^w_{m-1}) \nonumber
 \end{eqnarray}
 \noindent Output the decrypted message 
 \begin{equation}
  \hat{\mathcal{M}} = c_2\frac{{e}(\hat{d_1},U.c_0)}{{e}(\hat{d_2},c_1)}\nonumber
 \end{equation}
\noindent The proof of correctness of this scheme is presented in Appendix \ref{app_subsec:proofCCA}. Note that the overhead for the ciphertext, aggregate key, public parameters, and the private and public keys, remains unchanged. The main change from the original scheme is in the fact that decryption requires a randomization value $w\in\mathbb{Z}_q$.

\subsubsection{Claim C.1.} \textit{For a given $i\in\mathcal{S}$, the pair $(\hat{d_1},\hat{d_2})$ is chosen from the following distribution 
\begin{equation}
\left(\left({Y_{2^m-1}}\right)^{-1}.\left(PK.Y_i.g^{V_{SIG}}_{m-1}\right)^x,\left(g_{m-1}\right)^x\right)\nonumber
\end{equation}
\noindent where $x$ is uniformly randomly chosen from $\mathbb{Z}_q$.}\\
\noindent{\textbf{Proof.}} Similar to the proof in Appendix \ref{app_subsec:claim3.7}.\\

Once again, note that the distribution $(\hat{d_1},\hat{d_2})$ depends \emph{only on the data class $i$ for the message $\mathcal{M}$ to be decrypted} and is \emph{completely independent of the subset $\mathcal{S}$ used to encrypt it}. We next prove the non-adaptive CCA security of this scheme. 

\subsubsection{Theorem C.2.} \textit{Let \textbf{Setup}$'$ be the setup algorithm for a symmetric multilinear map, and let the decisional $(m,l)$-Multilinear Diffie-Hellman Exponent assumption holds for {SetUp}$'$. Then our proposed construction of KAC for $N'$ data classes presented in this section is non-adaptively CCA secure for $N'=\binom{m}{l}$, where each data class number $i<2^m-2$.}

\subsubsection{Proof.} Let $\mathcal{A}$ be a poly-time adversary such that $|Adv_{\mathcal{A},N'}-\frac{1}{2}| > \epsilon$ for the proposed KAC system parameterized with an identity space $\mathcal{ID}'$ of size $N'$, and $\epsilon$ is a non-negligible positive constant. We build an algorithm $\mathcal{B}$ that has advantage at least $\epsilon$ in solving the decisional $(m,l)$-MDHE problem for \textbf{Setup}$'$. $\mathcal{B}$ takes as input a random $(m,l)$-MDHE challenge tuple $(params',\{X_j\}_{j\in\{0,\cdots,m\}},V,Z)$, where:
\begin{itemize}
 \item $param'\leftarrow SetUp'(1^{\lambda},m+l-1)$
 \item $X_j=g^{\alpha^{(2^j)}}_{1}$ for $0\leq j \leq m$
 \item $V=g^{t'}_{l}$ for a random $t'\in\mathbb{Z}_q$ (where $q$ is a $\lambda$ bit prime)
 \item $Z$ is either $g^{t'\alpha^{(2^m-1)}}_{m+l-1}$ or a random element of $\mathbb{G}_{m+l-1}$.
\end{itemize}
\noindent $\mathcal{B}$ then proceeds as follows.\\

% \begin{enumerate}
\noindent \textbf{Commit:} $\mathcal{B}$ runs $\mathcal{A}$ and receives the set ${\mathcal{S}}^{*}$ of data classes that $\mathcal{A}$ wishes to be challenged on. $\mathcal{B}$ then randomly chooses a data class $i\in{\mathcal{S}}^{*}$ and provides it to $\mathcal{A}$.\\
 
\noindent \textbf{SetUp}: $\mathcal{B}$ should generate the public $param$, public key $PK$, the authentication key $U$, and the aggregate key $K_{\overline{{\mathcal{S}}^{*}}}$ and provide them to $\mathcal{A}$. Algorithm $\mathcal{B}$ first runs the $SigKeyGen$ algorithm to obtain a signature signing key $K^{*}_{SIG}$ and a corresponding verification key $V^{*}_{SIG} \in \mathbb{Z}_q$. The various items to be provided to $\mathcal{A}$ are generated as follows.
%  \vspace{-0.6mm}
\begin{itemize}
  \item $param$ is set as $(param'',\{X_i\}_{i\in\{0,\cdots,m\}})\nonumber$.
  \item $PK$ is set as $\left({g^u_{l}}\right)/{\left(Y_i.g{V^{*}_{SIG}}_{m-1}\right)}$ where $u$ is chosen uniformly at random from $\mathbb{Z}_q$. Note that this is equivalent to setting $msk=u-\alpha^i-V^{*}_{SIG}$, as required.
  \item $U$ is set as $g^{t}_{\mathbf{m}}$ where $t$ is again chosen uniformly at random from $\mathbb{Z}_q$.
  \item $\mathcal{B}$ then computes   
  \begin{equation}
   K_{\overline{{\mathcal{S}}^{*}}} = \prod_{v\notin{\mathcal{S}}^{*}}\frac{Y^{u}_{2^m-1-v}}{(Y_{2^m-1-v+i}).(g{V^{*}_{SIG}}_{m-1})}\nonumber
  \end{equation}
  \noindent Observe that $K_{\overline{{\mathcal{S}}^{*}}}=\prod_{v\notin{\mathcal{S}}^{*}}PK^{\alpha^{2^m-1-v}}$, as desired. Moreover, $\mathcal{B}$ is aware that $i\notin \overline{{\mathcal{S}}^{*}}$ (implying $i\neq v$), and hence has all the resources to compute $K_{\overline{{\mathcal{S}}^{*}}}$.  
\end{itemize}
 
\noindent Since the $g_{\mathbf{m}}$, $\alpha$, $u$, $t'$ and $t$ values are chosen uniformly at random, \emph{the public parameters and the public, private and authentication keys have an identical distribution to that in the actual construction}.\\

\noindent\textbf{Query Phase 1}: Algorithm $\mathcal{A}$ now issues decryption queries. Let $(\mathcal{C},v)$ be a decryption query $\mathcal{C}$ is obtained by $\mathcal{A}$ using some subset $\mathcal{S}$ containing $v$. However, $\mathcal{B}$ is not given the knowledge of $\mathcal{S}$. Let $\mathcal{C}=((c_0,c_1,c_2),\sigma,V_{SIG})$. Algorithm $\mathcal{B}$ first runs $Verify$ to check if the signature $\sigma$ is valid on $(c_0,c_1,c_2)$ using $V_{SIG}$. If invalid, $\mathcal{B}$ returns $\bot$. If $V_{SIG} = V^{*}_{SIG}$, $\mathcal{B}$ outputs a random bit $b\in\{0,1\}$ and \emph{aborts} the simulation. Otherwise, the challenger picks a random $x\in\mathbb{Z}_q$. It then sets
\begin{eqnarray}
 \hat{d_0}&=&Y^{(V_{SIG}-V^{*}_{SIG})}.Y_v.Y^{-1}_i\nonumber\\
 \hat{d'_0}&=&(Y_{v+1}/Y_{i+1})^{\frac{1}{(V_{SIG}-V^{*}_{SIG})}}\nonumber \\
 \hat{d_2}&=&g^{x}_{\mathbf{m}}.Y^{\frac{1}{(V_{SIG}-V^{*}_{SIG})}}_1\nonumber\\
 \hat{d_1}&=&\left(\hat{d_2}\right)^u.\left(\hat{d_0}\right)^x.\left(\hat{d'_0}\right) \nonumber 
\end{eqnarray}
\noindent Note that $\hat{d'_0}$ can be computed following Claim 4.1.b as $1\leq i,v \leq 2^m-3$. $\mathcal{B}$ responds with $K=c_2\frac{{e}(\hat{d_1},c_0.U)}{{e}(\hat{d_2},c_1)}$. 

\subsubsection{Claim C.3.} \textit{$\mathcal{B}$'s response is exactly as in a real attack scenario, that is, for some $x'$ chosen uniformly at random from $\mathbb{Z}_q$, we have} 
\begin{equation}
\hat{d_1} = \left({Y_{2^m-1}}\right)^{-1}.\left(PK.Y_v.g^{V_{SIG}}_{m-1}\right)^{x'}\hspace{2mm} \text{ and }\hspace{2mm} \hat{d_2} = g^{x'}_{m-1}\nonumber
\end{equation}

\noindent{\textbf{Proof.}} Similar to the proof in Appendix \ref{app_subsec:claim3.9}.\\

% 
% To see that this response is as in a real attack game, set $x'=x+\frac{\alpha}{(V_{SIG}-V^{*}_{SIG})}$ and observe that 
% 
% \noindent Furthermore, since $x$ is uniform in $\mathbb{Z}_q$, so is $x'$. 

\noindent Thus, by the result in Claim C.1, $\mathcal{B}$'s response is identical to \textbf{Decrypt}$(\mathcal{C},v,\mathcal{S},K_{\mathcal{S}},U)$, even though $\mathcal{B}$ does not possess the knowledge of the subset $\mathcal{S}$ used by $\mathcal{A}$ to obtain $\mathcal{C}$.\\
 
\noindent \textbf{Challenge}: $\mathcal{A}$ picks at random two messages $\mathcal{M}_0$ and $\mathcal{M}_1$ from the set of possible plaintext messages in $\mathbb{G}_{2\mathbf{m}}$, and provides them to $\mathcal{B}$. $\mathcal{B}$ randomly picks $b\in\{0,1\}$, and sets 
\begin{eqnarray}
 \mathcal{C}&=&(U^{-1}V,V^u,\mathcal{M}_b.Z) \nonumber\\
 \mathcal{C}^{*}&=&(\mathcal{C},Sign(\mathcal{C},K^{*}_{SIG}),V^{*}_{SIG})\nonumber
\end{eqnarray}
\noindent The challenge posed to $\mathcal{A}$ is $(\mathcal{C}^{*},\mathcal{M}_0,\mathcal{M}_1)$. It can be easily shown that when $Z=g^{t'\alpha^{(2^m-1)}}_{m+l-1}$ (i.e. the input to $\mathcal{B}$ is a valid $m$-MDHE tuple), then this is a valid challenge to $\mathcal{A}$ as in a real attack.\\

\noindent\textbf{Query Phase 2}: Same as in query phase 1.\\
 
\noindent \textbf{Guess}: The adversary $\mathcal{A}$ outputs a guess $b'$ of $b$. If $b' = b$, $\mathcal{B}$ outputs $0$ (indicating that $Z=g^{t'\alpha^{(2^m-1)}}_{m+l-1}$). Otherwise, it outputs $1$ (indicating that $Z$ is a random element in $\mathbb{G}_{m+l-1}$).\\

\noindent It can be easily proved that the probability that $\mathcal{B}$ aborts the simulation as a result of one of the decryption queries by $\mathcal{A}$ is less than $\epsilon_2$ (from the existential unforgability property of the signature scheme). We conclude that $\mathcal{B}$ has the same advantage $\epsilon$ as $\mathcal{A}$, which must therefore be negligible, as desired. This completes the proof of Theorem C.2. \hfill\qed
